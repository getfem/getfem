\documentclass{article}
\usepackage{amsmath}
\newcommand{\TL}{\ensuremath{\underline{L}}}
\newcommand{\TC}{\ensuremath{\underline{C}}}
\newcommand{\DU}{\ensuremath{\nabla u}}
\newcommand{\DV}{\ensuremath{\nabla v}}
\newcommand{\Dh}{\ensuremath{\nabla h}}
\newcommand{\DUt}{\ensuremath{\nabla u^t}}
\newcommand{\DVt}{\ensuremath{\nabla v^t}}
\newcommand{\Dht}{\ensuremath{\nabla h^t}}
\newcommand{\DTL}{\ensuremath{\nabla\TL}}
\newcommand{\DTLt}{\ensuremath{\nabla\TL^t}}
\newcommand{\TS}{\ensuremath{\hat{\hat{\sigma}}}}
\newcommand{\Wlaw}{\ensuremath{W}}
\newcommand{\dx}{\ensuremath{~\mathrm{dx}}}
\DeclareMathOperator{\Div}{div}
\DeclareMathOperator{\Det}{det}
\DeclareMathOperator{\Trace}{tr}
\begin{document}

$\Phi(x) = u(x) + x$, $\Phi$ is the position, $u$ is the displacement with respect to the initial position.

\begin{align}
\TL &= \frac{1}{2}\left(\nabla\Phi^t\nabla\Phi - I)\right)\\
&= \frac{1}{2}\left(\DUt\DU + \DUt + \DU\right)
\end{align}

In the case of linear elasticity, $\DUt\DU$ is neglected.

The so-called right Cauchy-Green strain tensor
\begin{align}
\TC &= \nabla\Phi^t\nabla\Phi &= 2\TL + I
\end{align}
is frequently used instead of $\TL$.

\subsection{Principal invariants and their derivatives}
$i_1,i_2,i_3$ are the invariants of orders $1,2$ and $3$:
\begin{align}
  i_1(\TL) &= \Trace\TL &i_1(\TC) &= 2\Trace\TL + 3\\
  i_2(\TL) &= \frac{\Trace\TL^2 - (\Trace\TL)^2}{2}\quad& i_2(\TC)&=4i_2(\TL)+4i_1(\TL)+3\\
  i_3(\TL) &= \Det\TL &i_3(\TC) &= 8i_3(\TL) + 4i_2(\TL) + 2i_1(\TL) + 1
\end{align}
Their derivatives in the direction $H$ are:
\begin{align}
  \frac{\partial i_1}{\partial\TL} &= I:H &= \Trace H\\
  \frac{\partial i_2}{\partial\TL} &= (i_1(\TL)I - \TL):H &= \Trace \TL:\Trace H - \TL:H\\
  \frac{\partial i_3}{\partial\TL} &= (i_2(\TL)I - i_1(\TL)\TL + \TL^2):H \\
                  &= i_3(\TL)(L^{-1}):H
\end{align}
and
\begin{align}
\frac{\partial (M^{-1})}{\partial M}(H) &= -M^{-1}HM^{-1}
\end{align}

\subsection{Potential elastic energy and its derivative}

The stress (in the reference configure, a.k.a. second Piola-Kirchhoff stress tensor) $\TS = \nabla\Phi^{-1}\sigma\nabla\Phi^{-t}~\Det \nabla\Phi$ is given by
\begin{align}
  \TS &= -\frac{\partial}{\partial\TL} \Wlaw(\TL)
\end{align}
where $\Wlaw$ is the strain energy of the material.

\begin{align}
  \mathcal{I}(u) &= \int_\Omega W(\TL(u))\dx\\
  (D\mathcal{I}(u))(v) &= \int_{\Omega} \frac{\partial W}{\partial\TL}(\TL(u)):(I+\DUt)\DV \dx
  \intertext{because}
  (D\TL(u))(v) &= \frac{1}{2}(\DUt\DV + \DVt\DU + \DVt + \DV)\\
               &= \frac{1}{2}(\DVt(I+\DU) + (I+\DUt)\DV)
               \intertext{and $A:B = A:(B+B^t)/2$ if A is symmetric, and $\TS$ is symmetric}
\end{align}



Another way to put it is
\begin{align}
-\Div \left((I+\DU)\TS\right) = f.
\end{align}

Integrating by parts, one obtains:
\begin{align}
  \int_\Omega(I + \DU)\TS : \DV \dx = l(v)
\end{align}

\subsection{Tangent matrix}
The displacement $u$ is fixed.


To obtain the tangent matrix, one subsitutes $u$ with $u+h$ 
\begin{align}
  \int_\Omega(I + \DU + \Dh)\TS(\TL(u)+\TL(h) + \frac{1}{2}(\Dht\DU+\DUt\Dh) : \DV \dx = l(v)
\end{align}
and considers the linear part w.r.t. $h$, which is
\begin{align}
  \int_\Omega\Dh\TS(\TL(u)) : \DV \dx +\\
  \int_\Omega (I+\DU)\frac{\partial}{\partial\TL}(\TL(u))\left(\frac{\Dh+\Dht+\Dht\DU+\DUt\Dh}{2}\right) : \DV \dx
\end{align}
which is symmetric w.r.t. $v$ and $h$. It can be rewritten as
\begin{align}
  \int_\Omega (I+\DU)\mathcal{A}((I+\DUt)\Dh):\DV~\dx
\end{align}
where $\mathcal{A}$ is the $3\times3\times3\times3$ tensor $\frac{\partial}{\partial\TL}(\TL(u))$.

\subsection{Material laws}
\subsubsection{Linearized: Hooke (small deformations)}
\begin{align}
\Wlaw &= \frac{\lambda}{2}i_1(\TL)^2 + \mu i_1(\TL^2)\\
\TS   &= \lambda i_1(\TL)I + 2\mu\TL\\
\frac{\partial\TS}{\partial\TL}(\TL ; H) &= \lambda i_1(H)I + 2\mu H
\end{align}
\subsubsection{Mooney-Rivlin}
Incompressible material.

\begin{align}
\Wlaw &= c_1(i_1(\TC) - 3) + c_2(i_2(\TC)-2)\\
      &= 2c_1i_1(\TL) + 4c_2(i_2(\TL)+i_1(\TL))
\intertext{with the additional constraint:}
  i_3(\TC) = 1
\end{align}
where $c_1$ and $c_2$ are given coefficients.
\begin{align}
  \TS &= (2c_1 + 4c_2)I + 4c_2(i_1(L)I - L)\\
  \frac{\partial\TS}{\partial\TL}(\TL ; H) &= 4c_2i_1(H)I - H
\end{align}

The incompressibility constraint $i_3(\TC) = 1$ is handled with a Lagrange multiplier $p$ (the pression):
\begin{align}
  -\int_{\Omega_0} p \TC^{-1}\Det\TC : (I+\DU^t)\DV \dx
\end{align}


Incompressibility Round 2:

constraint: $\sigma = -pI \Rightarrow \TS = p\nabla\Phi\nabla\Phi^{-T}\det\nabla\Phi$ 
\begin{align}
  1 - i_3(\nabla\Phi) &= 0 \\
  -\int_{\Omega_0} (\det\nabla\Phi  -1) w \dx &= 0
\end{align}

\begin{align}
 B &= -\int_{\Omega_0} p(\nabla\Phi)^{-T} \det \nabla\Phi : \nabla v \dx \\
 K &= \int_{\Omega_0} \left( p(\nabla\Phi)^{-T}(\nabla h)^{T}(\nabla\Phi)^{-T}\det\nabla\Phi : \nabla v \dx - 
   p(\nabla\Phi)^{-T}(\det \nabla\Phi(\nabla\Phi)^{-T}:\nabla h) : \nabla v \right) \dx\\
   &= \int_{\Omega_0} p(\nabla h^T\nabla\Phi^{-T}):(\nabla\Phi^{-1}\nabla v)\det\nabla\Phi dx - \int_{\Omega_0} p(\nabla\Phi^{-T}:\nabla h)(\nabla\Phi^{-T}:\nabla v)\det\nabla\Phi dx
\end{align}

\subsubsection{Ciarlet-Geymonat}
\begin{align}
\Wlaw &= \gamma_1i_1(\TL) + \frac{\lambda}{2}i_2(\TL) + 8ci_3(\TL) - \frac{\gamma_1}{2} \log \Det \TC
\end{align}

\end{document}